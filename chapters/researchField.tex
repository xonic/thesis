\chapter{Research Field}

\section{Communication and Information Retrieval with a Pen-based Meeting Support Tool}
Das Paper befasst sich mit dem prototypisch implementierten Meeting Support Tool "We-Met". Dieses verwendet ein stiftbasiertes Interface, das durch digitales Skizzieren der Benutzer die Kommunikation während Meetings fördern und die nachträgliche Auswertung der angefertigten Zeichnungen erleichtern soll.

\subsection{We-Met}
We-Met kann bei face-to-face Meetings als auch bei geographisch entfernten Meetings zusammen mit einer Telefonkonferenz eingesetzt werden. Jeder Teilnehmer benötigt einen Computer mit einem Bildschirm und einem stiftbasierten Eingabetablett. Die Geräte werden über ein lokales LAN Netzwerk verbunden. Das Interface bietet mehrere gemeinsame Bereiche für Skizzen und Zeichnungen, die nach jedem vollendeten Strich eines Benutzers aktualisiert werden. Somit sehen alle Teilnehmer die Zeichnungen der anderen nahezu in Echtzeit.

Alle Meetings werden aufgezeichnet und können während dessen oder nach Abschluss der Sitzung aufgearbeitet werden. Jeder Zeichenstrich wird mit einem Zeitstempel versehen, sodass der Zeichenprozess Schritt für Schritt vor- und zurückgespult werden kann. Außerdem erlaubt We-Met bestimmte zeitliche und räumliche Zustände der Skizzen mit Tags zu versehen, sodass sie leichter zwischen diesen hin und her springen können. Die Form der Tags ist an Markierungen angelehnt, die Personen auch in ihren echten Notizbüchern verwenden, denn sie können nicht nur aus Textelementen, sondern auch aus handgezeichneten Schnörkeln, Kügelchen oder Sternchen bestehen, wodurch das Anlegen von Tags einfacher und intuitiver wird.

\subsection{Studie: We-Met als Tool zur Kommunikation in Gruppen}
Schon während der Entwicklung von Prototypen ist es wichtig, das Konzept so früh wie möglich zu testen und evaluieren. We-Met wurde daher schon sehr bald in einer Studie mit potentiellen Nutzern getestet. Die Entwickler zogen drei Gruppen mit je drei Teilnehmern und eine Gruppe mit zwei Teilnehmern heran. Die Testpersonen erhielten die Aufgabe unter Gebrauch von We-Met einen Haushaltsroboter zu konzipieren, der Müll aufsammelt und in einen dafür vorgesehenen Behälter wirft. Viel mehr als ein finales Design zu schaffen, ging es darum, möglichst viele Ideen zu generieren und zu skizzieren.
\subsubsection{Resultate}
\begin{itemize}
	\item \textbf{Einfacher Zugang durch stiftbasiertes Interface}\\
	Alle Teilnehmer empfanden das stiftbasierte Interface einfach zu benutzen. Es fiel ihnen nicht schwer, zuzuhören oder zu reden und gleichzeitig zu schreiben. Das Arbeiten mit einer Tastatur hingegen erfordert bei vielen Personen einen zu hohen kognitiven Aufwand, um einer Diskussion noch genügend Aufmerksamkeit gewähren zu können. Aufgrund dieser Tatsache könnte das stiftbasierte Interface die Produktivität eines solchen Meetings erhöhen, da Teilnehmer mehrere Aktivitäten parallel durchführen können.
\end{itemize}